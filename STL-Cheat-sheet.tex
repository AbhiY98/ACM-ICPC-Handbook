\section{STL Cheat Sheet}

\Section{Algorithms}

\Topic{Non-modifying sequential operations}

\begin{tabular}{p{2cm}|p{5cm}|p{5.5cm}}
	funcName(s)	&	return type and arguements	&	Details	\\
	all\_of, any\_of, none\_of	& bool funcName(it first, it last, [UnaryPredicate])	&	Test for conditions in the given range	\\
	for\_each 	&	fn funcName(it first, it last, Function fn) &	executes function fn with arguements as elements in the range \\
	find, find\_if, find\_if\_not	&	it funcName(it first, it last, T val (or) Function fn)	&	returns iterator to first element which is equal to/satisfies fn/dissatisfies fn \\

\end{tabular}

\Topic{Modifiers}
% http://www.cplusplus.com/reference/algorithm/?kw=algorithm

\Topic{Partitions}

\begin{tabular}{p{2cm}|p{5cm}|p{5.5cm}}
    partition   &   iter partition (it first, it last, UnaryPredicate)  &   Reaaranges the range such that all elements which satisfy the predicate are before which do not, returns an iter to first which doesn't \\
    stable\_partition    &   Same as above   &   same as above but stable \\
\end{tabular}

\Topic{Sorting}

\begin{tabular}{p{2cm}|p{5cm}|p{5.5cm}}
    sort, stable\_sort   &   void funcName(it first, it second, [mycompare])   &   Sorts/stably sorts with/out mycompare function, mycompare should return true for a<b if sorting is required in increasing order \\
    partial\_sort    &   void partial\_sort(it first, it middle, it last, [mycompare])    &   partitions about the middle position (similar to partition used in median look-up) elements left to middle are sorted \\

\end{tabular}
